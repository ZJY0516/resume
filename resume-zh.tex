
% Chinese version
\documentclass[zh]{resume}

% Adjust icon size (default: same size as the text)
\iconsize{\Large}

% File information shown at the footer of the last page
\fileinfo{%
\faCopyright{} 朱江云 \hspace{0.5em}
  \creativecommons{by}{4.0} \hspace{0.5em}
  %\githublink{ZJY0516}{resume} \hspace{0.5em}
  \faEdit{} \today
}

\name{江云}{朱}


\keywords{BSD, Linux, Programming, Python, C, Shell, DevOps, SysAdmin}

% \tagline{\icon{\faBinoculars}} <position-to-look-for>}
% \tagline{<current-position>}

% Supported shapes: circular (default), square
% \photo[<shape>]{<width>}{<filename>}

\profile{
\mobile{199-8466-6173}
\email{jyz@smail.nju.edu.cn}
\github{ZJY0516} \\
\university{南京大学}
% \birthday{2003-05-16}
\homepage{https://riverclouds.net}
%\address{上海}
  % Custom information:
  % \icontext{<icon>}{<text>}
  % \iconlink{<icon>}{<link>}{<text>}
}

\begin{document}
\makeheader

%======================================================================
% Summary & Objectives
%======================================================================
% {\onehalfspacing\hspace{2em}%
% 南京大学物理专业本科生
% \par}

%======================================================================
\sectionTitle{教育背景}{\faGraduationCap}
%======================================================================
\begin{educations}
  \education%
  {2021.09}%
  {南京大学}%
  {物理学院}%
  {物理学}%
% \item 排名:18\%
\item 排名:18\%。修读课程:操作系统(90),计算机系统基础(85),形式语言与自动机等。
\item 保研至\link{http://oslab.ios.ac.cn/}{中科院软件所操作系统实验室}
\end{educations}


%======================================================================
\sectionTitle{个人项目}{\faCode}
%======================================================================
\begin{itemize}
  \item \link{https://github.com/ZJY0516/ics-nju}{\texttt{NEMU}}:
riscv32指令集的全系统模拟器,支持大部分指令,包括中断、异常等,可以解释执行riscv32的二进制文件,具有输入输出、上下文管理和虚拟内存管理的能力,并成功在上面运行了仙剑奇侠传。
\item \texttt{OS}:基于\link{https://git.nju.edu.cn/jyy/os-workbench}{os-workbench},实现了支持多CPU的内存管理、多线程、进程调度等功能。
\item \texttt{AlphaFold3推理优化}:使用Triton优化关键算子如Attention和GLU(Gated Linear Unit),提高推理速度。
\item \link{https://github.com/ZJY0516/nju-compiler-lab}{\texttt{Compiler}}:小型的类C语言(C\verb!--!)编译器,实现了包括词法分析、语法分析、语义分析、中间代码生成、目标代码生成等功能,能将C\verb!--!源代码转换成MIPS汇编代码。
\item \texttt{GEMM}:使用C语言和Intrinsic函数,利用矩阵分块、重排和SIMD指令集优化等方法优化矩阵乘法,提高计算效率,达到理论浮点性能的70\%。
\end{itemize}

%======================================================================
\sectionTitle{荣誉奖项}{\faAward}
%======================================================================
\begin{itemize}
  \item ASC世界大学生超级计算机竞赛二等奖
\item 北京大学超算比赛三等奖(37/869)
  \item 国家励志奖学金
  \item 全国大学生数学建模竞赛二等奖
\end{itemize}
% \begin{tabular}{L{0.45\linewidth}  L{0.5\linewidth}}
%   \hspace*{0.3em} \faAngleRight \, 全国大学生数学建模竞赛二等奖     & \faAngleRight \, ASC世界大学生超级计算机竞赛二等奖 \\
%   \hspace*{0.3em} \faAngleRight \, 北京大学超算比赛三等奖(37/558) & \faAngleRight \,国家励志奖学金
% \end{tabular}

%======================================================================
\sectionTitle{科研经历}{\faAtom}
%======================================================================
\begin{itemize}
  \item 在\link{http://css.ios.ac.cn/}{中科院软件所操作系统实验室}
处进行机器学习系统与加密数据库相关研究
\end{itemize}


%======================================================================
\sectionTitle{编程技能}{\faWrench}
%======================================================================
\begin{competences}
  \comptence{操作系统}{%
    \icon{\faLinux} Linux
  }
  \comptence{编程语言}{%
    \icon{\faPython}Python, C, C++, Shell, \LaTeX
  }
  \comptence{开发工具}{%
    \icon{\faGit*}Git, SSH, Make, Docker
  }
% \comptence{\icon{\faLanguage} 语言}{
%   \textbf{英语} --- 流畅阅读文献和技术资料
% }
\end{competences}


%======================================================================
\sectionTitle{学生工作}{\faUserFriends}
%======================================================================
\begin{itemize}
\item 担任\link{https://itxia.club}{南京大学IT侠互助协会}社长,举办过数次技术讨论沙龙,帮助同学维修电脑200余次。
\end{itemize}

% %======================================================================
% \sectionTitle{自我评价}{\faMarker}
% %======================================================================
% % \begin{itemize}
% %   \item 有良好的工程能力和编程能力
% %   \item 有自驱力和探索精神,善于学习新知识,对技术有浓厚兴趣
% % \item 有一定的领导能力和团队协作能力
% % \end{itemize}
% \begin{tabular}{L{0.45\linewidth}  L{0.5\linewidth}}
%   \hspace*{0.3em} \faAngleRight \, 有良好的工程能力和编程能力 & \faAngleRight \, 有自驱力和探索精神,对技术有浓厚兴趣 \\
% \end{tabular}
\end{document}
